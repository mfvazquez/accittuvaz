\documentclass[10pt,spanish,a4paper,openany,notitlepage]{article}
%-------------------------------------Paquetes-----------------------------------------------------------------------
\usepackage[spanish,es-tabla]{babel}  	% Traduce los textos a castellano
\usepackage[utf8]{inputenc}	% Permite escribir directamente áéíóúñ
\usepackage{t1enc}            	% Agrega caracteres extendidos al font
\usepackage{amsmath} 		%Permite imprimir mas opcciones matematicas
\usepackage{graphicx}		%Permite agregar imagenes al informe
\usepackage{multicol}  		%Permite dividir el texto en varias columnas
\usepackage{float} 		%Permite utilizar H para colocar las imagenes en un lugar especifico 
\usepackage{units}
\usepackage{circuitikz}
\usepackage{caption}
\usepackage{subcaption}
\usepackage{sidecap}
\usepackage{mathtools}
\usepackage{multirow} % Paquete para dividir las tablas en subtablas
\usepackage{booktabs} %estos 2 sirven para achicar la tabla
\usepackage{tabulary}
\usepackage{fancyhdr} % encabezado
\usepackage{textcomp} % para usar ° con el comando \textdegree
\usepackage{anysize}		%Permite modificar los margenes del documento
\usepackage{abstract} % paquete para el resumen del articulo
 
%---------------------------------------Configuraciones de pagina----------------------------------------------
\marginsize{2.5cm}{2.5cm}{1cm}{1cm}

\pagestyle{fancy}
\fancyhf{}
\lhead{
66.25 - \textsc{Dispositivos Semiconductores}\\ 
2\textsuperscript{do} Cuatrimestre de 2014
}
\rhead{\includegraphics[width=3cm]{imagenes/FIUBA_ALTA.jpg}}
\rfoot{Página \thepage}

%---------------------------------------Definiciones propias---------------------------------------------------------
\newcommand{\oiint}{\displaystyle\bigcirc\!\!\!\!\!\!\!\!\int\!\!\!\!\!\int} %Integral doble cerrada

\DeclarePairedDelimiter\abs{\lvert}{\rvert}%
\DeclarePairedDelimiter\norm{\lVert}{\rVert}%
% Swap the definition of \abs* and \norm*, so that \abs
% and \norm resizes the size of the brackets, and the 
% starred version does not.
\makeatletter
\let\oldabs\abs
\def\abs{\@ifstar{\oldabs}{\oldabs*}}
%
\let\oldnorm\norm
\def\norm{\@ifstar{\oldnorm}{\oldnorm*}}
\makeatother
%--------------------------------------------------------------------------------------------------------------------------------


\makeatletter
\let\ps@plain\ps@fancy 
\makeatother

% lo siguiente es para borrar el titulo del resumen y que no ocupe espacio:
 \AtBeginDocument{%
 \renewcommand{\abstractname}{}%
 }
\renewcommand{\absnamepos}{empty} % originally center
 

\begin{document}
\title{\textbf{TP N\textdegree4: Diseño y construcción de un mini-amplificador de audio}}
\author{
  Accifonte, Franco - 93799\\
  \texttt{franco.accifonte@gmail.com}  
  \and
  Iturria, Germán  - 86270 \\
  \texttt{german.iturria@gmail.com}
  \and
   Vázquez, Matías - 91523\\
  \texttt{mfvazquez@gmail.com}
}
\date{2 de diciembre de 2014}
\maketitle

\begin{abstract} %Resumen
\emph{}
\end{abstract}

\section{Desarrollo}


\begin{figure}[H]
\centering
\begin{circuitikz}[american]\shorthandoff{>}
\draw 
(5,1) node[npn](npn){BC548B}
(0,0)  node[ground]{} to [sV, l=$v_{in}$] (0,1) 
to [R, l=$R_S$] (2,1)
to [C, l=$C_{in}$] (3,1)-- (npn.base)
(2,3) node[ground]{} to [V, l=$V_{CC}$] (2,4)
to [short] (5,4)
to [R, l=$R_C$] (5,2) -- (npn.collector)
(3.5,4) to [R, l=$R_B$] (3.5,1)
(5,0)node[ground]{} -- (npn.emitter) 
(8,0)  node[ground]{} to [R, l_=$R_L$] (8,1.77) 
to [C, l_=$C_{out}$] (5.5,1.77) -- (npn.collector)
;\end{circuitikz}
\caption{Circuito }
\label{circuito:amplificador}
\end{figure}


\begin{figure}[H]
\centering
\begin{circuitikz}[american]\shorthandoff{>}
\draw 
(5,1) node[npn](npn){BC548B}
(2,0) node[ground]{} to [V, l=$V_{CC}$] (2,1)
to [R, l=$R_B$] (4.5,1) --  (npn.base)
(2,1) to [short] (2,2.5)
to [R, l=$R_C$] (5,2.5) -- (npn.collector)
(5,0)node[ground]{} -- (npn.emitter) 
(7,0)  node[ground]{} to [R, l_=$R_L$] (7,1.77) 
to [short] (5.5,1.77) -- (npn.collector)
;\end{circuitikz}
\caption{Circuito }
\label{circuito:amplificador_dc}
\end{figure}

\begin{figure}[H]
\centering
\begin{circuitikz}[american]\shorthandoff{>}
\draw 
(4,0)  node[ground]{} to [R, l=$R_B$] (4,2) 
(2,0) to [R, l=$r_{\pi}$] (2,2)
(0,0) to [sV, l=$v_{in}$] (0,2)
to [R,l=$R_S$] (2,2)
to [short] (4,2)
(0,0) to [short] (10,0)
(6,2) to [I,l_=$g_m\, v_{be}$] (6,0)
(8,0) to [R,l=$r_o$] (8,2)
(10,0) to [R,l=$R_C$] (10,2)
(6,2) to [short] (10,2)
to [short, -o, l=$v_{out}$] (10.5,2)
;\end{circuitikz}
\caption{Circuito }
\label{circuito:amplificador_ac}
\end{figure}


\end{document}
