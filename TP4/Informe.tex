\documentclass[10pt,spanish,a4paper,openany,notitlepage]{article}
%-------------------------------------Paquetes-----------------------------------------------------------------------
\usepackage[spanish,es-tabla]{babel}  	% Traduce los textos a castellano
\usepackage[utf8]{inputenc}	% Permite escribir directamente áéíóúñ
\usepackage{t1enc}            	% Agrega caracteres extendidos al font
\usepackage{amsmath} 		%Permite imprimir mas opcciones matematicas
\usepackage{graphicx}		%Permite agregar imagenes al informe
\usepackage{multicol}  		%Permite dividir el texto en varias columnas
\usepackage{float} 		%Permite utilizar H para colocar las imagenes en un lugar especifico 
\usepackage{units}
\usepackage{circuitikz}
\usepackage{caption}
\usepackage{subcaption}
\usepackage{sidecap}
\usepackage{mathtools}
\usepackage{multirow} % Paquete para dividir las tablas en subtablas
\usepackage{booktabs} %estos 2 sirven para achicar la tabla
\usepackage{tabulary}
\usepackage{fancyhdr} % encabezado
\usepackage{textcomp} % para usar ° con el comando \textdegree
\usepackage{anysize}		%Permite modificar los margenes del documento
\usepackage{abstract} % paquete para el resumen del articulo
\usepackage{amssymb}
%---------------------------------------Configuraciones de pagina----------------------------------------------
\marginsize{2.5cm}{2.5cm}{1cm}{1cm}

\pagestyle{fancy}
\fancyhf{}
\lhead{
66.25 - \textsc{Dispositivos Semiconductores}\\ 
2\textsuperscript{do} Cuatrimestre de 2014
}
\rhead{\includegraphics[width=3cm]{imagenes/FIUBA_ALTA.jpg}}
\rfoot{Página \thepage}

%---------------------------------------Definiciones propias---------------------------------------------------------
\newcommand{\oiint}{\displaystyle\bigcirc\!\!\!\!\!\!\!\!\int\!\!\!\!\!\int} %Integral doble cerrada

\DeclarePairedDelimiter\abs{\lvert}{\rvert}%
\DeclarePairedDelimiter\norm{\lVert}{\rVert}%
% Swap the definition of \abs* and \norm*, so that \abs
% and \norm resizes the size of the brackets, and the 
% starred version does not.
\makeatletter
\let\oldabs\abs
\def\abs{\@ifstar{\oldabs}{\oldabs*}}
%
\let\oldnorm\norm
\def\norm{\@ifstar{\oldnorm}{\oldnorm*}}
\makeatother
%--------------------------------------------------------------------------------------------------------------------------------


\makeatletter
\let\ps@plain\ps@fancy 
\makeatother

% lo siguiente es para borrar el titulo del resumen y que no ocupe espacio:
 \AtBeginDocument{%
 \renewcommand{\abstractname}{}%
 }
\renewcommand{\absnamepos}{empty} % originally center
 

\begin{document}
\title{\textbf{TP N\textdegree4: Diseño y construcción de un mini-amplificador de audio}}
\author{
  Accifonte, Franco - 93799\\
  \texttt{franco.accifonte@gmail.com}  
  \and
  Iturria, Germán  - 86270 \\
  \texttt{german.iturria@gmail.com}
  \and
   Vázquez, Matías - 91523\\
  \texttt{mfvazquez@gmail.com}
}
\date{2 de diciembre de 2014}
\maketitle

\begin{abstract} %Resumen
\emph{En el siguiente trabajo se realiza el diseño y construcción de un
mini-amplificador de audio empleando un amplificador de source común.}
\end{abstract}

\section{Especificaciones}

Se requiere construir un circuito simple que amplifique la tensión que
genera un micrófono de bobina móvil de $600\, \unit{\Omega}$, considerando
que mantenga una señal de $25\, \unit{m\widehat{V}}$, para que pueda ser digitalizada
por un convertor digital {\bf MAX1393}. El grabador se alimenta con
una batería de $1.5\,\unit{V}$ ($1200\, \unit{mAh}$ de carga), y debe
minimizarse tanto como sea posible el consumo de potencia del sistema.
Las condiciones de diseño del amplificador son:

\begin{itemize}
\item La salida del amplificador debe permitir la máxima excursión
posible entre $0\, \unit{V}$ y $5\, \unit{V}$, que es el rango de entrada
del conversor A/D.
\item La resistencia de salida del amplificador debe ser menor a $5\, \unit{k\Omega}$.
\item La potencia de contínua debe ser tal que permita su uso por más de
$24\, \unit{hs}$.
\item Se debe obtener la mayor ganancia posible, respetando lo anterior.
\item El amplificador debe ser emisor común, es decir, debe utilizarse
el transistor {\bf TBJ BC548}.
\item Se debe considerar que la resistencia que presenta el conversor A/D
al amplificador es mayor a $1\, \unit{M\Omega}$.
\end{itemize} 

\section{Diseño del amplificador}

Se definió el circuito mostrado en la figura \ref{circuito:amplificador}.
Utilizando los siguientes valores.
\begin{itemize}
\item $R_S = 600\, \unit{\Omega}$
\item $R_L = 1\, \unit{M\Omega}$
\item $V_{CC} = 1.5\, \unit{V}$
\item $v_s = 25\, \unit{m\widehat{V}}$ 
\item $C_{in} = C_{out} = 1\, \unit{\mu F}$
\end{itemize}

\begin{figure}[H]
\centering
\begin{circuitikz}[american]\shorthandoff{>}
\draw 
(5,1) node[npn](npn){BC548B}
(0,0)  node[ground]{} to [sV, l=$v_s$] (0,1) 
to [R, l=$R_S$] (2,1)
to [C, l=$C_{in}$] (3,1)-- (npn.base)
(2,3) node[ground]{} to [V, l=$V_{CC}$] (2,4)
to [short] (5,4)
to [R, l=$R_C$] (5,2) -- (npn.collector)
(3.5,4) to [R, l=$R_B$] (3.5,1)
(5,0)node[ground]{} -- (npn.emitter) 
(8,0)  node[ground]{} to [R, l_=$R_L$] (8,1.77) 
to [C, l_=$C_{out}$] (5.5,1.77) -- (npn.collector)
;\end{circuitikz}
\caption{Circuito amplificador}
\label{circuito:amplificador}
\end{figure}

Para los calculos se utilizaron los parámetros obtenidos del transistor 1
en el Trabajo Práctico N\textdegree2 ya que es el transistor que se utilizará
al armar el circuito.

\begin{itemize}
\item $\beta = 361$
\item $V_{th} = 26.95\, \unit{mV}$
\item $V_{A} = 36.05\, \unit{V}$
\item $V_{BE} = 0.7 \, \unit{V}$
\end{itemize}

\subsection{Punto de polarización}

Considerando a los capacitores como circuitos abieirtos y 
pasivando la fuente de señal $v_s$ obtenemos el circuito mostrado en 
la figura \ref{circuito:amplificador_dc}.

\begin{figure}[H]
\centering
\begin{circuitikz}[american]\shorthandoff{>}
\draw 
(5,1) node[npn](npn){BC548B}
(2,0) node[ground]{} to [V, l=$V_{CC}$] (2,1)
to [R, l=$R_B$] (4.5,1) --  (npn.base)
(2,1) to [short] (2,2.5)
to [R, l=$R_C$] (5,2.5) -- (npn.collector)
(5,0)node[ground]{} -- (npn.emitter)
(5,2.5) to [short, -o, l=$V_{OUT}$] (5.5,2.5) 
;\end{circuitikz}
\caption{Circuito amplificador en DC}
\label{circuito:amplificador_dc}
\end{figure}

Se consideró la ecuación \ref{eq:VOUT} para los calculos ya que es
cuando se obtiene la máxima eficiencia.

\begin{equation}
V_{OUT} = \frac{V_{CC}}{2}
\label{eq:VOUT}
\end{equation}

Asumiendo que el TBJ está en MAD y aplicando mallas al circuito de 
la figura \ref{circuito:amplificador_dc} obtenemos las siguientes 
ecuaciones:

\[ \displaystyle V_{CC} - V_{BE} = I_B R_B\]

\[ \displaystyle V_{CC} - V_{OUT} = I_C R_C\]

Y teniendo cuenta que $I_C = \beta I_B$ y la ecuación numero \ref{eq:VOUT}
obtenemos las siguientes relaciones de $R_C$ y $R_B$ respecto a $I_C$:

\begin{equation}
R_C = \frac{V_{CC}}{2 I_C}
\label{eq:RC}
\end{equation}

\begin{equation}
R_B = \frac{(V_{CC} - V_{BE}) \beta}{I_C}
\label{eq:RB}
\end{equation}

Luego se verificó que el punto Q este en zona de MAD teniendo en cuenta
la ecuación \ref{eq:RC}:

\[ \displaystyle V_{CE} = V_{CC} - I_C R_C = V_{CC} - \frac{V_{CC}}{2} = \frac{V_{CC}}{2} = 0.75\, \unit{V} > V_{CE_{sat}} \approx 0.2\, \unit{V} \]

Finalmente, debido a que la potencia de contínua debe ser tal que permita su uso
por más de $24\, \unit{hs}$ y la batería cuenta con $1200\, \unit{mAh}$
de carga, podemos obtener la cota maxima de la corriente $I_{C}$.

\[ \displaystyle I_{C} \leqslant \frac{1200\, \unit{mAh}}{24\, \unit{hs}} \]

Por lo tanto:

\begin{equation}
I_{C} \leqslant 50\, \unit{mA}
\label{eq:IC_max}
\end{equation}


\subsection{Modelo de pequeña señal}

Pasivando las fuentes de tensión continua y reemplazando el transistor
por su modelo equivalente de pequeña señal, obtenemos el circuito
de la figura \ref{circuito:amplificador_ac}.

\begin{figure}[H]
\centering
\begin{circuitikz}[american]\shorthandoff{>}
\draw 
(5,0)  node[ground]{}
(4,0) to [R, l=$R_B$] (4,2) 
(2,2) to [R, l_=$r_{\pi}$, v^=$v_{be}$] (2,0)
(0,0) to [sV, l=$v_s$] (0,2)
to [R,l=$R_S$] (2,2)
to [short] (4,2)
(0,0) to [short] (10,0)
(6,2) to [I,l_=$g_m v_{be}$] (6,0)
(8,0) to [R,l=$r_o$] (8,2)
(10,0) to [R,l=$R_C$] (10,2)
(6,2) to [short] (10,2)
to [short, -o, l=$v_{out}$] (10.5,2)
;\end{circuitikz}
\caption{Circuito amplificador en AC}
\label{circuito:amplificador_ac}
\end{figure}

\begin{equation}
r_\pi = \frac{V_{th} \beta}{I_C}
\label{eq:rpi}
\end{equation}

\begin{equation}
g_m = \frac{I_C}{V_{th}}
\label{eq:gm}
\end{equation}

\begin{equation}
r_o = \frac{V_A}{I_C}
\label{eq:ro}
\end{equation}

Del circuito \ref{circuito:amplificador_ac} se obtiene $v_{out}$

\[ \displaystyle v_{out} = -g_m v_{be} (r_o // R_C) \]

Luego la ganancia de tensión sin carga es

\[ \displaystyle A_{vo} = \frac{v_{out}}{v_{be}} = -gm (r_o // R_C) \]

Reemplazando las ecuaciones \ref{eq:RC}, \ref{eq:gm} y \ref{eq:ro} se obtiene

\[ \displaystyle A_{vo} = -\frac{I_C}{V_{th}} \frac{\frac{V_A V_{CC}}{2 I_C^2}}{\frac{V_A}{I_C} + \frac{V_{CC}}{2 I_C}} \]

Simplificando se obtiene

\begin{equation}
\displaystyle A_{vo} = -\frac{V_A V_{CC}}{V_{th} (2 V_A + V_{CC})}
\label{eq:Avo}
\end{equation}

Reemplazando los valores obtenemos la ganancia $A_{vo}$

\[ \displaystyle A_{vo} = -28.26 \]

A continuación se analizaron las tres causas de distorsión para obtener las
cotas de $I_C$

\subsubsection{Distorsión por alinealidad}

Para que se pueda utilizar el modelo de pequeña señal del TBJ, se
debe cumplir que $v_{be} \leqslant 10\,\unit{m\widehat{V}}$, ya que este valor es una 
cota máxima en la que una vez superado se pierde la linealidad de las 
ecuaciones utilizadas y se observa la distorsión en la salida con 
respecto a la señal de entrada, por lo tanto se calcula el valor  
máximo de $r_{\pi}$ para que al conectar el microfono en la entrada 
(con su resistencia interna), la caída de tensión $v_{be}$ no supere este valor 
maximo.

Primero se obtiene la cota de $v_{out}$ para poder compararla con las
cotas de las otras distorsiónes. Y recordando que $v_{be}$ y $v_{out}$
están en contra fase:

\[ \displaystyle v_{be;max} = \frac{v_{out;min}}{A_{vo}} = 10\,\unit{m\widehat{V}}\]

Se despeja y calcula $v_{out;min}$

\[ \displaystyle v_{out;min} = v_{be;max} A_{vo} = 10\,\unit{m\widehat{V}} (-28.26) = -282.6 \,\unit{m\widehat{V}}\]

Como $v_{out}$ es una señal sinusoidal $v_{out;max} = -v_{out;min}$ entonces
$v_{out;max} = 282.6 \,\unit{m\widehat{V}}$. 
Por lo tanto la cota de $v_{out}$ para la distorsión por alinealidad será:

\begin{equation}
v_{out} \leqslant 282.6\, \unit{mV}
\label{eq:vout_alinealidad}
\end{equation}


Luego asumiendo que $r_\pi << R_B$ se obtiene $(r_\pi // R_B) \approx r_\pi$.
Con la aproximación anterior resolvemos el divisor de tensión del circuito
de la figura \ref{circuito:amplificador_ac} entre las resistencias $r_\pi$
y $R_S$.

\[ \displaystyle v_{be} = v_s \frac{r_\pi}{R_S + r_\pi} \]

Como $v_{be} \leqslant 10\, \unit{m\widehat{V}}$:

\[ \displaystyle v_s \frac{r_\pi}{R_S + r_\pi} \leqslant 10\,\unit{m\widehat{V}}\]

Despejando $r_\pi$ y reemplazando datos se obtiene:

\begin{equation}
r_\pi \leqslant 400\, \unit{\Omega}
\label{eq:rpi_cota}
\end{equation}

Finalmente reemplazando la ecuación \ref{eq:rpi} en la inecuación \ref{eq:rpi_cota}
se obtiene:

\[ \displaystyle \frac{V_{th} \beta}{I_C} \leqslant 400\, \unit{\Omega} \]

Despejando $I_C$ y reemplazando datos obtenemos:

\begin{equation}
I_C \geqslant 24.32\, \unit{mA}
\label{eq:IC_alinealidad}
\end{equation}

\subsubsection{Distorsión por corte}

Para $v_s$ demasiado negativa el transistor entra en corte, entonces
toda la corriente de señal circula por la resistencia $R_C$.

Se calcula la cota maxima de $v_{out}$, reemplazando la ecuación \ref{eq:VOUT}

\[ \displaystyle v_{out;max} = V_{CC} - V_{OUT} = 1.5\, \unit{V} - 0.75\, \unit{V} = 0.75\, \unit{V} \]

Por lo tanto la cota de $v_{out}$ para la distorsión por corte será:

\begin{equation}
v_{out} \leqslant 750\, \unit{mV}
\label{eq:vout_corte}
\end{equation}

La cota maxima de $v_{out}$ por alinealidad es menor a la cota por corte. Por
lo que evitando la distorsion por alinealidad se estará evitando la
distorsión por corte.

\subsubsection{Distorsión por saturación}

Para $v_s$ muy positiva el TBJ entra en régimen de saturación. El caso
límite tolerable es:

\[ \displaystyle v_{out;max} = V_{OUT} - V_{sat} = 0.75\, \unit{V} - 0.3\, \unit{V} = 0.45\, \unit{V} \]

Por lo tanto la cota de $v_{out}$ para la distorsión por saturación será:

\begin{equation}
v_{out} \leqslant 450\, \unit{mV}
\label{eq:vout_corte}
\end{equation}

Nuevamente la cota maxima de $v_{out}$ por alinealidad es menor a la cota por saturación.
Por lo tanto evitando distorsión por alinealidad se estará también evitando
distorsión por saturación.

\subsection{Elección de $I_C$, $R_C$ y $R_B$}

La cota mínima de $I_C$ es debido a la distorsión por alinealidad y
su cota maxima es debido a que el amplificador pueda ser usado por más
de $24 \unit{hs}$.

\begin{equation}
24.32\, \unit{mA}\leqslant I_C \leqslant 50\, \unit{mA}
\label{eq:IC_cotas}
\end{equation}

Mediante las ecuaciónes \ref{eq:RC} y \ref{eq:RB} obtenemos las cotas para $R_C$
y $R_B$, respectivamente.

\begin{equation}
30.84\, \unit{\Omega}\leqslant R_C \leqslant 15\, \unit{\Omega}
\label{eq:RC_cotas}
\end{equation}

\begin{equation}
11.88\, \unit{k\Omega}\leqslant R_B \leqslant 5.78\, \unit{k\Omega}
\label{eq:RB_cotas}
\end{equation}


\subsection{Cálculo teórico}

\subsection{Dispersión de $\beta$}

\subsection{Comparación con source común}

\section{Simulación del circuito}

\section{Mediciones del circuito}

\section{Conclusiones}

\end{document}
